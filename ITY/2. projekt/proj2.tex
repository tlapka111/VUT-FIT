\documentclass[11pt, a4paper, twocolumn]{article}

\usepackage[utf8]{inputenc}
\usepackage[czech]{babel}
\usepackage[IL2]{fontenc}
\usepackage{times}
\usepackage[left=15mm, top=25mm, text={180mm, 250mm}]{geometry}
\usepackage{amsmath}
\usepackage{amsthm}
\usepackage{amsfonts}
\usepackage{hyperref}

\theoremstyle{definition}
\newtheorem{definition}{Definice} 

\newtheorem{sentence}{Věta}

\begin{document}

\begin{titlepage}
    \begin{center}
    	{\textsc{\Huge Fakulta informačních technologií\\[0,4em]
    	Vysoké učení technické v Brně}}\\
    	\vspace{\stretch{0.382}}
    	{\LARGE Typografie a publikování – 2. projekt\\[0,3em]
		Sazba dokumentů a matematických výrazů}
		\vspace{\stretch{0.618}}
    \end{center}
	\LARGE 2018 \hfill Tomáš Tlapák (xtlapa00)
\end{titlepage}

\section*{Úvod}
V této úloze si vyzkoušíme sazbu titulní strany, matematických
vzorců, prostředí a dalších textových struktur obvyklých
pro technicky zaměřené texty (například rovnice (\ref{eq:rovnice})
nebo Definice \ref{def:definice} na straně \pageref{def:definice}). Rovněž si vyzkoušíme používání odkazů \verb|\ref| a \verb|\pageref|. \par

Na titulní straně je využito sázení nadpisu podle optického středu s využitím zlatého řezu. Tento postup byl
probírán na přednášce. Dále je použito odřádkování se
zadanou relativní velikostí 0.4em a 0.3em. \par

\section{Matematický text}

Nejprve se podíváme na sázení matematických symbolů
a~výrazů v plynulém textu včetně sazby definic a vět s~využitím
balíku \verb|amsthm|. Rovněž použijeme poznámku pod
čarou s použitím příkazu \verb|\footnote|. Někdy je vhodné
použít konstrukci \verb|${}$|, která říká, že matematický text
nemá být zalomen. \par

\begin{definition}\label{def:definice}
Turingův stroj \emph{(TS) je definován jako šestice tvaru $M$ = ($Q, \Sigma, \Gamma, \delta, q_0, q_1$) kde}:
\end{definition}

\begin{itemize}
	\item $Q$ \emph{je konečná množina} vnitřních (řídicích) stavů,
	\item $\Sigma$ \emph{je konečná množina symbolů nazývaná} vstupní
abeceda, $\Delta \notin \Sigma$,
	\item $\Gamma$ \emph{je konečná množina symbolů}, $\Sigma \subset \Gamma,\  \Delta \in \Gamma$, \emph{nazývána} pásková abeceda,
	\item{ $\delta: (Q$\textbackslash $\{q_F\})\times \Gamma\rightarrow Q\times(\Gamma\cup\{L,R\})$, \emph{kde} $L, R\notin \Gamma$, \emph{je parciální} přechodová funkce,}
	\item $q_0$ \emph{je} počáteční stav, $q_0\in Q$ \emph{a}
	\item $q_F$ \emph{je} koncový stav, $q_F \in Q$.	
\end{itemize}

Symbol $\Delta$ značí tzv. \emph{blank} (prázdný symbol), který se vyskytuje na místech pásky, která nebyla ještě použita
(muže ale být na pásku zapsán i později). \par

\emph{Konfigurace pásky} se skládá z nekonečného řetězce, který reprezentuje obsah pásky a pozice hlavy na tomto řetězci. Jedná se o prvek množiny $\{\gamma\Delta^\omega \mid \gamma \in \Gamma^\ast\}\  \times \mathbb{N}$.\footnote{Pro libovolnou abecedu $\Sigma$ je $\Sigma^\omega$ množina všech \emph{nekonečných} řetězců nad $\Sigma$, tj. nekonečných posloupností symbolů ze $\Sigma$. Pro připomenutí: $\Sigma^\ast$ je množina všech \emph{konečných} řetězců nad $\Sigma$.} \emph{Konfiguraci pásky} obvykle zapisujeme jako $\Delta xyz\underline{z}x\Delta...$ (podtržení značí pozici hlavy).\emph{ Konfigurace stroje} je pak dána stavem řízení a konfigurací pásky. Formálně se jedná
o prvek množiny $Q \times \{\gamma\Delta^\omega \mid \gamma \in \Gamma^\ast\} \times \mathbb{N}$.

\subsection{Podsekce obsahující větu a odkaz}

\begin{definition}\label{def:rovnice2}
Řetězec $\omega$ nad abecedou $\Sigma$ je přijat TS \emph{M
jestliže M při aktivaci z počáteční konfigurace pásky} $\underline{\Delta}w\Delta...$ \emph{a počátečního stavu} $q_0$ \emph{zastaví přechodem do koncového stavu} $q_F$, \emph{tj}. $(q_0, \Delta w \Delta^\omega, 0) \underset{M}{\overset{\ast}{\vdash}}(q_F, \gamma, n)$\emph{ pro nějaké} $\gamma \in~ \Gamma^\ast$ \emph{a} $n \in \mathbb{N}$. \par
\emph{Množinu} $L(M) = \{w \mid w$ \emph{je přijat TS M}\} $\subseteq \Sigma^\ast$ \emph{nazýváme} jazyk přijímaný TS \emph{M}.
\end{definition}

Nyní si vyzkoušíme sazbu vět a důkazů opět s použitím balíku \verb|amsthm|.

\begin{sentence}
\emph{Třída jazyků, které jsou přijímány TS, odpovídá
rekurzivně} vyčíslitelným jazykům.
\end{sentence}

\begin{proof}[Důkaz]
V důkaze vyjdeme z Definice \ref{def:definice} a \ref{def:rovnice2}.
\end{proof}

\section{Rovnice a odkazy}
Složitější matematické formulace sázíme mimo plynulý
text. Lze umístit několik výrazu na jeden řádek, ale pak je
třeba tyto vhodně oddělit, například příkazem \verb|\quad|. \par

\begin{center}
$\sqrt[i]{x_i^3}$ kde $x_i$ je $i$-té sudé číslo $y_i^{2.y_i} \neq y_i^{y_i^{y_i}}$
\end{center}

V rovnici (\ref{eq:rovnice}) jsou využity tři typy závorek s různou
explicitně definovanou velikostí.

\begin{eqnarray} \label{eq:rovnice}
x &=& \bigg\lbrace \Big( \big[ a+b \big] * c \Big)^d \oplus 1 \bigg\rbrace \\
y &=& \lim_{x\to\infty} \frac{\sin^2x + \cos^2x}{\frac{1}{\log_{10} x}}\nonumber
\end{eqnarray}

V této větě vidíme, jak vypadá implicitní vysázení limity $\lim_{n\to\infty} f(n)$ v normálním odstavci textu. Podobně je to i s dalšími  symboly jako $\sum_{i=1}^{n} 2^i$ či $\bigcup_{A\in B}A$. V~ případě
vzorců $\lim\limits_{x\to\infty} f(n)$ a $\sum\limits_{i=1}^n 2^i$ jsme si vynutili méně úspornou sazbu příkazem \verb|\limits|.

\begin{eqnarray}
\int\limits_{a}^{b}f(x) \ \mathrm{d}x &=& - \int_{b}^{a}g(x)\  \mathrm{d}x \\
\overline{\overline{A \vee B}} &\Leftrightarrow & \overline{\overline{A}\wedge \overline{B}}
\end{eqnarray}

\section{Matice}

Pro sázení matic se velmi často používá prostředí \verb|array| a závorky (\verb|\left|, \verb|\right|) \par

\[ \left( \begin{array}{ccc}
a+b & \widehat{\xi + \omega} & \hat{\pi} \\
\vec{a} & \overleftrightarrow{AC} & \beta \end{array} \right)= 1 \Longleftrightarrow \mathbb{Q} = \mathbb{R} \] 

\[
\mbox{\textbf{A}} = \left\| \begin{array}{cccc}
 a_{11} & a_{12} & \ldots & a_{1n} \\
 a_{21} & a_{22} & \ldots & a_{2n} \\ 
 \vdots & \vdots & \ddots & \vdots \\
 a_{m1} & a_{m2} & \ldots & a_{mn} 
 \end{array} \right\| 
 = \left| \begin{array}{cc} 
 t & u \\ 
 v & w 
 \end{array} \right|
 = tw\!-\!uv
\]

Prostředí \verb|array| lze úspěšně využít i jinde.

\[ 
\left(\! \begin{array}{c} 
 n  \\ 
 k 
 \end{array} \!\right) 
 = \left\{ \begin{array}{ll}
 \frac{n!}{r!(n-r)!} & \text{pro } 0 \leq k \leq n \\
 0 & \text{pro } k < 0\  \text{nebo }k > n
\end{array} \right.
\] 
 
 \section{Závěrem}

V případě, že budete potřebovat vyjádřit matematickou
konstrukci nebo symbol a nebude se Vám dařit jej nalézt
v samotném \LaTeX u, doporučuji prostudovat možnosti balíku
maker  \AmS-\LaTeX.

\end{document}